% Options for packages loaded elsewhere
\PassOptionsToPackage{unicode}{hyperref}
\PassOptionsToPackage{hyphens}{url}
%
\documentclass[
]{article}
\usepackage{lmodern}
\usepackage{amssymb,amsmath}
\usepackage{ifxetex,ifluatex}
\ifnum 0\ifxetex 1\fi\ifluatex 1\fi=0 % if pdftex
  \usepackage[T1]{fontenc}
  \usepackage[utf8]{inputenc}
  \usepackage{textcomp} % provide euro and other symbols
\else % if luatex or xetex
  \usepackage{unicode-math}
  \defaultfontfeatures{Scale=MatchLowercase}
  \defaultfontfeatures[\rmfamily]{Ligatures=TeX,Scale=1}
  \setmainfont[]{Times New Roman}
  \setmonofont[]{Lucida Console}
\fi
% Use upquote if available, for straight quotes in verbatim environments
\IfFileExists{upquote.sty}{\usepackage{upquote}}{}
\IfFileExists{microtype.sty}{% use microtype if available
  \usepackage[]{microtype}
  \UseMicrotypeSet[protrusion]{basicmath} % disable protrusion for tt fonts
}{}
\makeatletter
\@ifundefined{KOMAClassName}{% if non-KOMA class
  \IfFileExists{parskip.sty}{%
    \usepackage{parskip}
  }{% else
    \setlength{\parindent}{0pt}
    \setlength{\parskip}{6pt plus 2pt minus 1pt}}
}{% if KOMA class
  \KOMAoptions{parskip=half}}
\makeatother
\usepackage{xcolor}
\IfFileExists{xurl.sty}{\usepackage{xurl}}{} % add URL line breaks if available
\IfFileExists{bookmark.sty}{\usepackage{bookmark}}{\usepackage{hyperref}}
\hypersetup{
  hidelinks,
  pdfcreator={LaTeX via pandoc}}
\urlstyle{same} % disable monospaced font for URLs
\usepackage[margin=1in]{geometry}
\usepackage{graphicx}
\makeatletter
\def\maxwidth{\ifdim\Gin@nat@width>\linewidth\linewidth\else\Gin@nat@width\fi}
\def\maxheight{\ifdim\Gin@nat@height>\textheight\textheight\else\Gin@nat@height\fi}
\makeatother
% Scale images if necessary, so that they will not overflow the page
% margins by default, and it is still possible to overwrite the defaults
% using explicit options in \includegraphics[width, height, ...]{}
\setkeys{Gin}{width=\maxwidth,height=\maxheight,keepaspectratio}
% Set default figure placement to htbp
\makeatletter
\def\fps@figure{htbp}
\makeatother
\setlength{\emergencystretch}{3em} % prevent overfull lines
\providecommand{\tightlist}{%
  \setlength{\itemsep}{0pt}\setlength{\parskip}{0pt}}
\setcounter{secnumdepth}{5}
\usepackage{fancyhdr, lastpage}
\pagestyle{fancy}
% \usepackage{graphicx}
% \graphicspath{{figs/}}
\usepackage{enumitem}
\usepackage{setspace}
\usepackage{verbatim}
\usepackage{hanging}
\usepackage{multirow}
\usepackage{float}
\usepackage{lscape}
\usepackage{tabulary}
\usepackage{tabularx}
\usepackage{tcolorbox}
\usepackage{array}
\newcolumntype{H}{>{\setbox0=\hbox\bgroup}c<{\egroup}@{}}

\usepackage{hyperref}
\hypersetup{  
   colorlinks,   
   citecolor=blue,
   linkcolor=blue,
   urlcolor=blue,
}

\usepackage{color, colortbl}
\definecolor{Gray}{gray}{0.9}

\renewcommand{\headrulewidth}{0pt}

\chead{}
\rhead[p. \thepage\ of\ \pageref*{LastPage}]{p. \thepage\ of\ \pageref*{LastPage}}
\lhead{}

\fancyfoot{}
\fancyfoot[L]{\textcolor{gray}{EDUC 7310-01}}
\fancyfoot[C]{\textcolor{gray}{Fall 2020}}
\fancyfoot[R]{\textcolor{gray}{Aaron R. Baggett, Ph.D.}}


\author{}
\date{\vspace{-2.5em}}

\begin{document}

\makeatletter
\setlength{\@fptop}{0pt}
\makeatother

\begin{figure}[t!]
  \centering
  \includegraphics[keepaspectratio, width=0.15\textwidth]{../Figs/Seal-Black}
\end{figure}


\begin{center}
{\LARGE{\bf{EDUC 7310: Research I Design and Methods}}}\\
% \vspace{.5mm}
% {\Large{\bf{Dissertation I}}}\\
{\small{University of Mary Hardin-Baylor}}\\
\vspace{.15in}
{\large{Fall 2020}}
\end{center}

\thispagestyle{empty}

\vspace{3mm}

\vspace{.25in}

\section{Concact Information}

\textbf{Instructor:} Aaron R. Baggett, Ph.D.\\
\textbf{Email:}
\texttt{\href{mailto:abaggett@umhb.edu}{abaggett@umhb.edu}}\\
\textbf{Office Phone:} (254) 295-4553\\
\textbf{Office Location:} Wells 140\\
\textbf{Office Hours:} MWF: 9:00 AM--11:00 AM; MW: 1:00 -- 3:00 PM; TR:
1:00 PM -- 3:00 PM, and by
\texttt{\href{mailto:abaggett@umhb.edu}{appointment}}

\begin{tcolorbox}
[width=\linewidth, sharp corners=all, colback=white!95!red]
NOTE: All student meetings for the Fall 2020 semester will occur via Dr. Baggett's personal \texttt{\href{https://umhb.zoom.us/j/5393191651}{Zoom ID}}
\end{tcolorbox}

\subsection{Description}

The purpose of Research I is to establish the foundation for doctoral
level research. The course offers an overview of both quantitative and
qualitative methods and designs with an emphasis on critically
evaluating research. Other topics include general data collection and
analysis. Students will develop a problem statement, literature review,
and potential research method regarding a self-selected topic in
educational research.

\subsection{Meeting}

\textbf{Dates:}

Students enrolled in EDUC 7310: Research I will meet together a total of
five times throughout the Fall 2020 semester between 1:30 PM--5:30 PM in
the Parker Academic Center (PAC) room 224.

\begin{enumerate}
\def\labelenumi{\arabic{enumi}.}
\tightlist
\item
  Saturday, September 12, 2020, PAC 224
\item
  Saturday, October 03, 2020, PAC 224
\item
  Saturday, October 24, 2020, PAC 224
\item
  Saturday, November 14, 2020, PAC 224
\item
  Saturday, December 05, 2020, PAC 224
\end{enumerate}

\textbf{Course Website:}
\texttt{\href{https://mycourses.umhb.edu/courses/23592}{myCourses}}

\newpage
\subsection{Advanced Academic Activity}

Doctoral courses contain appropriate advanced academic activity
reflected in the areas of content, process, and product. The advanced
activity is facilitated through the dimension of critical thinking,
synthesis and integration of materials, depth of engagement of
materials, and contribution to scholarship. The purpose of advanced
academic activity is to demonstrate a higher level of sophistication and
to emphasize separation from masters level courses.

\subsection{Course Objectives}

Upon completion of this course, you should be able to:

\begin{enumerate}
\def\labelenumi{\arabic{enumi}.}
\tightlist
\item
  Understand the basic quantitative foundations needed to both consume
  and conduct doctoral level educational and social science research.
\item
  Communicate and present a problem statement, literature review, and
  potential research method(s) regarding a self-selected topic in
  educational research in oral and written forms using appropriate
  technical terms for both a technical and non-technical audience.
\end{enumerate}

\subsection{Student Learning Objectives}

Upon completion of each course module, you should be able to:

\textbf{1. Module 1: The Process of Conducting Research Using
Quantitative Approaches I {[}Creswell and Guetterman (2019), Chs.
1--2{]}}

\begin{enumerate}
\def\labelenumi{\arabic{enumi}.}
\tightlist
\item
  Describe the six steps in the process of research
\item
  Identify the type of research designs associated with quantitative and
  qualitative research.
\item
  Define a research problem and explain its importance.
\item
  Distinguish between a research problem and other parts of the research
  process.
\end{enumerate}

\textbf{2. Module 2: The Process of Conducting Research Using
Quantitative Approaches II {[}Creswell and Guetterman (2019), Chs.
3--4{]}}

\begin{enumerate}
\def\labelenumi{\arabic{enumi}.}
\tightlist
\item
  Identify the six steps in conducting a literature review.
\item
  Describe why purpose statements, research questions, and hypotheses
  are important.
\item
  Produce quantitative purpose statements, research questions, and
  hypotheses.
\end{enumerate}

\textbf{3. Module 3: Methods for Collecting Quantitative Data
{[}Creswell and Guetterman (2019), Ch. 5{]}}

\begin{enumerate}
\def\labelenumi{\arabic{enumi}.}
\tightlist
\item
  State the five steps in the process of quantitative data collection.
\item
  Identify how to select participants for a study.
\item
  Describe procedures for quantitative data collection.
\end{enumerate}

\textbf{4. Module 4: Analyzing and Interpreting Quantitative Data
{[}Creswell and Guetterman (2019), Ch. 6{]}}

\begin{enumerate}
\def\labelenumi{\arabic{enumi}.}
\tightlist
\item
  Identify the steps in the process of analyzing and interpreting
  quantitative data
\item
  Identify the procedures for analyzing your data.
\item
  Learn how to interpret and report the results of data analyses.
\end{enumerate}

\textbf{5. Module 5: Introduction to Quantitative Research Designs
{[}Creswell and Guetterman (2019), Chs. 10, 11, 12{]}}

\begin{enumerate}
\def\labelenumi{\arabic{enumi}.}
\tightlist
\item
  Identify the key characteristics of experimental designs.
\item
  Describe the steps in conducting an experiment.
\item
  Describe the key characteristics of correlational designs.
\item
  Describe how to construct and administer survey questionnaires and
  instruments.
\end{enumerate}

\subsection{Credit Hour(s)}

For online, hybrid, and other nontraditional modes of delivery, credit
hours are assigned based on learning outcomes that are equivalent to
those in a traditional course setting; forty-five (45) hours of work by
a typical student for each hour of credit.

\subsection{Readings}

Students are required to obtain a copy of the following required
textbook.

\begin{hangparas}{.4in}{1}
Creswell, J. W., \& Guetterman, T. C. (2019). {\em{Educational research: Planning, conducting, and evaluating quantitative and qualitative research}} (6th ed.). Pearson.
\end{hangparas}

All other assigned readings, if applicable, will be provided on the
\texttt{\href{https://mycourses.umhb.edu/courses/23592}{course website}}
under the
\texttt{\href{https://mycourses.umhb.edu/courses/23592/files/folder/Readings}{Readings}}
tab.

\subsection{Recommended Software}

Any data and statistical analyses in EDAD 7310 will be conducted using
IBM\textsuperscript{\textregistered}~SPSS\textsuperscript{\textregistered}~Statistics.
During class meetings, students may access local
SPSS\textsuperscript{\textregistered}~installations on campus computers.
However, you may consider purchasing at least a 6-month license to
download and install SPSS\textsuperscript{\textregistered}~on your own
computer(s). Below are a few online educational software retailers who
offer heavily discounted versions of
IBM\textsuperscript{\textregistered}~SPSS\textsuperscript{\textregistered}~Statistics
for students.

\emph{Note}: As of \today, the latest version number is 27 However,
versions 23, 24, 25, and 26 should also be compatible for all data
analyses conducted throughout this course. Keep in mind previous
versions sell for the same price as the current version. Prior to
purchasing SPSS\textsuperscript{\textregistered}~be sure you are
selecting the versions compatible with your particular computer's
operating system (i.e., Windows, macOS, etc.).
IBM\textsuperscript{\textregistered}~SPSS\textsuperscript{\textregistered}~Statistics
is not compatible with iOS, iPadOS, or Chrome OS.

\textbf{How to purchase SSPS:}

\begin{tcolorbox}
[width=\linewidth, sharp corners=all, colback=white!95!red]
NOTE: IBM\textsuperscript{\textregistered}\ SPSS\textsuperscript{\textregistered}\ Statistics Standard GradPack will be sufficient for EDUC 7310. No need to purchase the Premium GradPack.
\end{tcolorbox}

\begin{enumerate}
\def\labelenumi{\arabic{enumi}.}
\tightlist
\item
  \texttt{\href{https://onthehub.com/spss/}{OnTheHub}}

  \begin{enumerate}
  \def\labelenumii{\arabic{enumii}.}
  \tightlist
  \item
    Select \textbf{Students} buy now.
  \item
    Scroll down until you find your computer's operating system (i.e.,
    Windows, macOS, etc.) and your desired rental duration (i.e., 06- or
    12-month).
  \item
    Add to Cart.
  \end{enumerate}
\item
  \texttt{\href{https://www.journeyed.com/products/IBM+SPSS/IBM+SPSS+Statistics}{journeyEd}}

  \begin{enumerate}
  \def\labelenumii{\arabic{enumii}.}
  \tightlist
  \item
    Scroll to approximately one-third of the way down the page and look
    for \textbf{IBM SPSS Statistics Standard Grad Pack 27.0 Academic}.
  \item
    Select the link applicable to your operating system.
  \item
    Note: only 12-month licenses are available from journeyEd.
  \end{enumerate}
\end{enumerate}

\subsection{Academic Integrity}

UMHB's policy on Classroom Expectations and Ethics will be strictly
upheld in this course. If you have not read it and all subsequent
sections, it is your responsibility to do so. You may find it online
here:
\href{http://catalog.umhb.edu/2019-2020/Graduate-Catalog/Classroom-Expectations-and-Ethics}{\texttt{Classroom Expectations and Ethics}}.
The omnibus policy outlines University requirements concerning Christian
citizenship, students' responsibilities, class attendance, academic
decorum, and academic integrity.

\subsection{Disabled Student Services and Accomodations}

It is the student's responsibility to request disability accommodations.
Students requesting an accommodation for a disability, must contact the
UMHB
\href{http://cths.umhb.edu/disability}{\texttt{Counseling, Testing \& Health Services}}
as early as possible in the term.
\href{http://catalog.umhb.edu/en/2019-2020/Graduate-Catalog}{\texttt{The Course Catalog}},
\href{http://students.umhb.edu/student-handbook}{\texttt{Student Handbook}}
and \href{https://go.umhb.edu/}{\texttt{UMHB website}} provide more
details regarding the process by which accommodation requests will be
reviewed.

For more information, please contact:

\textbf{Blayne Alaniz, Director of Student Disability Services and
Testing Services}\\
UMHB Box 8437\\
900 College Street\\
Belton, Texas 76513\\
Office: (254) 295-4739\\
Fax: (254) 295-4196\\
Email: \texttt{\href{mailto:balaniz@umhb.edu}{balaniz@umhb.edu}}

\subsection{Course Structure}

All assignments and other coursework are completed individually.
However, during and between certain class meetings you may either be
assigned to or asked to form small groups in order to collaborate on
data analysis projects and/or presentation(s) You will be guided through
the following course learning modules. See Section 1.5 for corresponding
student learning outcomes per learning module.

\subsubsection{Learning Modules}

EDAD 7310 is divided into four (4) learning modules:

\begin{enumerate}
\def\labelenumi{\arabic{enumi}.}
\tightlist
\item
  The Process of Conducting Research Using Quantitative Approaches I
\item
  The Process of Conducting Research Using Quantitative Approaches II
\item
  Methods for Collecting Quantitative Data
\item
  Analyzing and Interpreting Quantitative Data
\end{enumerate}

\subsection{Course Communication}

\subsubsection{Email}

Most all course communication outside of class will take place via
email. I will routinely email you course updates and announcements to
your UMHB-assigned email address. Thus, you should check your email
frequently. Likewise, due to the nature of this class and the
corresponding assignments, you will likely need to contact me with
questions. I am committed to responding as quickly as possible to your
questions via email. As a result, you can expect me to respond, on
average, within several hours of your email---often sooner. However, in
some circumstances, a personal visit during office hours or other
scheduled appointment may be more efficient than email. You are welcome
to call me on my office line: (254) 295-4553. This can be an even more
efficient method for quick troubleshooting inquiries.

\subsubsection{Remind}

There may be occasions when alerting you to course-related updates may
be most effective in real-time. In these situations, I will communicate
with you through a free, safe, and one-way messaging service called
Remind. To sign up for these alerts, text \texttt{@educ7310} to 81010
and follow the instructions. If you have trouble with this method, try
texting \texttt{@educ7310} to (254) 296-8301. Additionally, although not
likely, there may be extenuating circumstances which require me to delay
and/or cancel a class or other meeting.

\section{Course Requirements}

\subsection{Individual and Team Assignments}

\subsubsection{Reverse Engineering a Journal Article}

The purpose of this assignment is to read and examine critically all
sections of a peer-reviewed journal article. As a student of educational
research, it is imperative that you gain familiarity and comfort with
the structure/purpose of the scientific literature. A detailed
submission template and grading rubric are available
\texttt{\href{https://bit.ly/31bLEDx}{here}}.

For each critique, you will select one quantitative research article
published within the last five years in a peer-reviewed journal. You are
free to select any article from any reputable, peer-reviewed journal
under the following conditions:

\begin{enumerate}
\def\labelenumi{\arabic{enumi}.}
\tightlist
\item
  The topic of research and theoretical framework are \emph{unrelated}
  to your planned dissertation
  research.\footnote{The purpose of forcing you to branch outside of your own topic/theoretical framework is to allow you to, hopefully, experience different approaches to your own planned quantitative methodology. For example, various disciplines use a variety of nomenclature to describe the elements of research methods and design. Assume you have committed to learning how to play classical piano. All you listen to, practice, and perform are arrangements from the classical greats. However, imagine the perspective you might gain by listening to a jazz pianist improvise? The point is to expose you to various ways in which your methodology is implemented in other "genres" of research outside of you own. The more familiar you can be with your methodology now, the better off you will be both during your oral qualifying exam as well as your dissertation proposal presentation.}
\item
  However, the author(s) utilized quantitative methodology similar to or
  identical to that of your own planned research, as you currently
  understand it.
\end{enumerate}

Journal article critiques are due on the following dates:

\begin{enumerate}
\def\labelenumi{\arabic{enumi}.}
\tightlist
\item
  Sunday, September 27, 2020
\item
  Sunday, October 11, 2020
\end{enumerate}

\subsubsection{Special Topics Team Presentations}

This assignment consists of small groups of students presenting one or
more special topics from a section(s) of assigned readings from the
respective module. Teams should prepare a thorough lecture with
accompanying slides, handouts, etc. You should use no fewer than three
additional academic resources, not counting the textbook, to supplement
your presentation. Lectures should be organized, rigorous, and
comprehensive. You should assume the burden of responsibility for
providing your peers everything you can in order to ensure they have as
complete an understanding about your assigned topic(s) as you and your
partner. They will do the same.

See \texttt{\href{https://bit.ly/2EfvCzF}{Special Topics Presentations}}
under Files in the course Canvas page for your team's assigned topics.

Team numbers and student pairs were generated using a random sampling
permutation method. Based on the team number, chapter assignments and
dates were implemented by the instructor.

\begin{table}[H]
\begin{center}
\caption{Special Topics Team Presentation Dates}
\vspace{3mm}
\begin{tabular}{llll}
\hline
\textbf{Team} & \textbf{Students} & \textbf{Topics} & \textbf{Presentation Date} \\
\hline
\multirow{3}{*}{1} & Stephanie Bermudez & \multirow{2}{*}{Ch. 3} & \multirow{2}{*}{Saturday, October 03, 2020} \\
& Gerard Cortez &  &  \\
& Dena Sempe &  &  \\
\hline
\multirow{3}{*}{2} & Kayla Abshire & \multirow{3}{*}{Ch. 5} & \multirow{3}{*}{Saturday, October 24, 2020} \\
& Deborah Gilbertson &  &  \\
& Erika Gutierrez &  &  \\
\hline
\multirow{3}{*}{3} & Cassandra Chavira & \multirow{3}{*}{Ch. 6} & \multirow{3}{*}{Saturday, November 14, 2020} \\
& Charlotte Conner &  & \\
& Joannie Caraballo-Lopez &  & \\
\hline
\multirow{3}{*}{4} & Terre Evans & \multirow{3}{*}{Ch. 12} & \multirow{3}{*}{Saturday, December 05, 2020} \\
& Erica Hummel &  &  \\
& Tatiana Czarnecki &   & \\
\hline
\end{tabular}
\end{center}
\end{table}

\subsubsection{Chapter 1 in a Nutshell}

Your culminating assignment/project in EDAD 7310 will be comprised of an
abridged, rough sketch version of your dissertation's introduction
section (chapter 1). In chapter 1, you detail, in a sense, the
foundations of and evidence for your particular research problem, an
outline of your research questions and corresponding hypotheses, and any
terms that need defining.

The following pages contain the eight sections you will address in the
Chapter 1 in a Nutshell assignment. Each section contains a brief
description of how you should address each section. In brief, the eight
sections you will address are:

\begin{enumerate}
\def\labelenumi{\arabic{enumi}.}
\tightlist
\item
  Research Problem
\item
  Background of the Problem
\item
  Research Questions
\item
  Hypotheses
\item
  Variables
\item
  Sample/Population
\item
  Definition of Terms
\item
  Limitations
\end{enumerate}

I realize this assignment may feel a little daunting for some of you.
Not to worry. Again, the more familiar you can be with your planned
dissertation research now, the better off you will be both during your
oral qualifying exam as well as your dissertation proposal presentation.

Chapter 1 in a Nutshell is due \textbf{Friday, December 04, 2020.}

\subsection{Grade Calculation}

\subsubsection{Individual and Team Performance}

Table 3 below lists all assignments, their point value, and proportion
of weighted total. See Table 4 for final grade calculation and letter
grade distribution.

\begin{table}[H]
\centering
\caption{Individual Assignments and Point Values}
\vspace{3mm}
\label{points}
\begin{tabular}{lllrlrr}
\hline
\bf{Assignment} & \bf{\em{n}} &  & \bf{Points} &  & \bf{Total} & \bf{Prop.} \\
\hline
Journal Article Reviews & 2 & $\times$ & 50 & = & 100 & .25\\
Special Topics Team Presentations & 1 & $\times$ & 100 & = & 100 & .25\\
Chapter 1 in a Nutshell & 1 & $\times$ & 100 & = & 100 & .50 \\
\multicolumn{4}{r}{\bf{Individual Performance Total}} & {\bf{=}} & {\bf{300}} & {\bf{1.00}} \\
\hline
\end{tabular}
\label{points}
\end{table}

\subsubsection{Final Grade Calculation}

All course grades will be posted in the gradebook in myCourses. All
point totals and proportional weights listed in Table 2 are reflected in
myCourses. Thus, your current grade in myCourses should reflect your
actual grade. Table 4 below describes the point range required to
achieve a given letter grade.

\begin{table}[H]
\begin{center}
\caption{Final Grade Point Range Requirements}
\label{finalgrades}
\vspace{3mm}
\begin{tabular}{cccc}
\hline
\bf{Grade} & \bf{Point Range} & \bf{Percentage} & \bf{Grade Points}\\
\hline
A & 270.00 -- 300.00 & 90 -- 100 & 4.0\\ 
B & 240.00 -- 267.00 & 80 -- 89  & 3.0\\ 
C & 210.00 -- 237.00 & 70 -- 79  & 2.0\\ 
D & 180.00 -- 207.00 & 60 -- 69  & 1.0\\ 
F & 000.00 -- 177.00 & 00 -- 59  & 0.0\\
\hline
\end{tabular}
\end{center}
\end{table}

\section{Policies}

\subsection{Attendance}

Your regular attendance and participation in this course is expected. I
will record and maintain attendance records for each student. Attendance
is worth 5\% of your final grade. In other words, if you attend 100\% of
the scheduled class meetings you will earn the complete 5\% attendance
total. Any University- or otherwise-excused absence will not count
toward this total. At the conclusion of the semester, the percentage of
class meetings you attended will be multiplied by 0.05 to obtain your
attendance grade.

\subsection{Late Work}

All assignments are considered late if submitted after the date and time
specified in the syllabus and/or course website. This policy will be
enforced in the event that assignment deadlines are revised during the
course of the term. Assignments submitted late will result in a penalty
of 20 percentage points per day.\textbackslash{}

For example, if an assignment is due on March 22, 2020 and is submitted
within 24 hours of the due date and time that assignment will result in
an automatic deduction of 20 percentage points from the assignment raw
score. In other words, if you submit an assignment worth 10 points on
March 23, 2020, and the assignment was originally due March 22, 2020,
and you score a 9.5/10, then your new score would be:

\begin{equation}
9.5 - (9.5)(0.20) \times 100 = 7.6.
\end{equation}

Assignments submitted more than five calendar days late will receive a
grade of zero. To ensure fairness, this policy will be strictly
enforced. Exceptons are made at the discretion of the instructor and may
include, but are not limited to:

\begin{enumerate}
\def\labelenumi{\arabic{enumi}.}
\tightlist
\item
  Death in the immediate family (parent, spouse, sibling, child)
\item
  Unforeseeable medical emergency affecting yourself, your spouse, or
  your child (e.g., automobile accident, major sickness, et al.).
\item
  Participation in an official UMHB-sponsored event
\end{enumerate}

\emph{Note}: Routine medical appointments or clinical visits related to
minor illnesses do not qualify as an unforeseeable medical emergency.
Likewise, conflicts with a work schedule or trips not related to
official UMHB events do not qualify for assignment absolution.

\section{Disclaimer}

Syllabus is subject to change at instructor's discretion.

\newpage

\section{Course Calendar}

\begin{tabularx}{\linewidth}{XcXr}
\hline
\textbf{Module} & \textbf{Week} & \textbf{Date} & \textbf{Reading} \\
\hline

The Process of Conducting Research Using Quantitative Approaches I & 1 & Saturday, September 12, 2020 & Chs. 01--02 \\
 &  &  &  \\
 
The Process of Conducting Research Using Quantitative Approaches II & 2 & Saturday, October 03, 2020 & Chs. 03--04 \\
 &  &  & \\
 
Methods for Collecting Quantitative Data & 3 & Saturday, October 24, 2020 & Ch. 05 \\
 &  &  & \\
 
Analyzing and Interpreting Quantitative Data & 4 & Saturday, November 14, 2020 & Ch. 06 \\
 &  &  &  \\
 
Introduction to Quantitative Research Designs & 5 & Saturday, December 05, 2020 & Chs. 10--12 \\
\hline
\end{tabularx}

\newpage
\section{Calendar of Due Dates}

\begin{tabularx}{\linewidth}{p{4cm}Xr}
\hline
\textbf{Month} & \textbf{Date} & \textbf{Due} \\
\hline
\multirow{1}{*}{September} & Saturday, September 27, 2020 & Journal Article Review \#1 \\
\arrayrulecolor{gray}\hline
&  &  \\

\multirow{3}{*}{October} & Saturday, October 03, 2020 & Team 1 Presentation \\
& Sunday, October 11, 2020 & Journal Article Review \#2 \\
& Saturday, October 24, 2020 & Team 2 Presentation \\
\arrayrulecolor{gray}\hline
&  &  \\

\multirow{1}{*}{November} & Saturday, November 14, 2020 & Team 3 Presentation \\
\arrayrulecolor{gray}\hline
&  &  \\

\multirow{2}{*}{December} & Friday, December 04, 2020 & Chapter 1 in a Nutshell \\
& Saturday, December 05, 2020 & Team 4 Presentation \\
\arrayrulecolor{gray}\hline
&  &  \\

\end{tabularx}

\newpage
\section{CruFlex and Other Pandemic-Related Details}

For both semesters of the 2020-2021 academic year, UMHB will utilize
three approaches to class offerings:

\begin{enumerate}
\item A hybrid-flexible approach, called CRUflex, will be utilized for most class offerings. CRUflex courses will be set up to run face-to-face, online synchronously, and online asynchronously at the same time. Students will have the choice to attend any or all of these formats each week.
\item Some courses and programs, called non-CRUflex, will be excluded from components of the CRUflex model. Excluded courses may include classes such as clinicals, internships, activity courses, and courses/programs where specific accreditation requirements apply. Non-CRUflex courses will not be able to accommodate all three modalities simultaneously. Thus, there may be a mandatory face-to-face component or restrictions for online participation.
\item Traditional online courses will be offered as well. Online courses do not have an on campus component.
\end{enumerate}

Flexibility is at the heart of this new approach to classroom
instruction. For CRUflex courses, you may opt to do any of the
following:

\begin{enumerate} 
\item Take a course in a modified face-to-face manner that allows for social distancing
\item Take a course online for the entire semester 
\item Begin the semester remotely and attend face-to-face later in the semester
\item Begin the semester face-to-face and complete the course remotely
\item Choose a combination of modalities to fit your current needs.
\end{enumerate}

We will adopt a classroom model that divides most classes into multiple
cohorts. Students will attend classes face-to-face on certain days and
remotely at other times. The amount of time for each modality may be
different for each course depending on classroom capacity, number of
students in the class, and social distancing requirements. For the days
in which you are not in the classroom, you can attend remotely during
the same time as the class or even at a later time.

This hybrid model will go into effect for almost all courses, with
accommodations made for some classes such as clinicals, internships,
activity courses, and courses/programs where specific accreditation
requirements apply. More information about those courses will be coming
to you from your deans.

For additional information about CRUflex, please click here for the
\texttt{\href{https://bit.ly/3fJpoEW}{CRUflex FAQ}}.

\newpage

\begin{center}
\color{violet}{\bf{\large{The Cru Health and Safety Pledge}}}
\end{center}

\vspace{.25cm}

As a member of the UMHB community, to responsibly help mitigate the
potential spread of COVID-19, I promise to:

\textbf{CARE FOR MYSELF AND OTHERS:}

\begin{itemize}
\item Read and comply with the Safe Return to Campus plan, which includes training materials and health/safety protocols. 
\item Follow all UMHB health and safety protocols 
\item Conduct daily self-screening. 
\item Stay home and not enter campus facilities if I feel sick. 
\item Stay home and not enter campus facilities if I have been exposed to someone who has tested positive for COVID-19. 
\item Properly wear a face covering when in common areas of campus, including all classrooms. 
\item Wash or sanitize my hands often. 
\item Follow social distancing protocols (at least 6 feet of distance and limitations on numbers of individuals gathering in a common location) both on and off campus, not making assumptions about who may be more vulnerable to contracting this illness. 
\item Keep my clothing, belongings, personal spaces and shared common spaces clean, and not share personal items such as cell phones, eating utensils and water bottles with others, which could spread the COVID-19 virus.
\end{itemize}

\textbf{RESPECT OUR CAMPUS AND SURROUNDING COMMUNITY:}

\begin{itemize}
\item Follow all directions given by university officials and displayed on university signage. 
\item Be respectful and responsive when others remind me of these health and safety protocols. 
\item Pay attention to and observe national, state and local directives. 
\item Remember that not everyone is affected the same by COVID-19. By complying with COVID-19 health guidelines, I will help those who are most vulnerable to stay safe. 
\end{itemize}

\textbf{REPORT}

\begin{itemize}
\item Stay home and immediately notify Dr. Brandon Skaggs at (214) 704-1168 or Michael Burns at (405) 308-7336 or \texttt{\href{mailto:student.covid@umhb.edu}{student.covid@umhb.edu}} should I develop any of these symptoms: cough, shortness of breath or difficulty breathing, chills, repeated shakes with chills, muscle pain, headache, sore throat, loss of taste or smell, diarrhea, feeling feverish or a measured temperature of 100 F; or if I have known close contact with someone who is lab-confirmed positive to have COVID-19. 
\item I understand that contacting other faculty or staff does not fulfill my duty to immediately report to Dr. Skaggs or Mr. Burns.
\end{itemize}

I acknowledge that this Promise is a condition of my ability to
participate in the 2020-21 academic year and utilize university
facilities. My failure to comply may lead to immediate removal from
classes, from campus and/or the inability to use certain facilities.
Violations of this Promise will be referred, reviewed and adjudicated in
accordance with the procedure outlined in
\texttt{\href{https://go.umhb.edu/students/student-handbook}{UMHB’s Code of Student Conduct}}.

\end{document}
